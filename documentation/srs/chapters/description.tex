\section{Overall Description}
\subsection{Product Perspective}
	a-bec is a self-contained product that has no connections to other software.	

\subsection{Product Functions}
\begin{itemize}
	\item Creating / viewing ads for flats and houses
	\item General search functionality with specific criteria
	\item Scheduling visits
	\end{itemize}

\subsection{User Classes and Characteristics}
	\subsubsection{User}
		The general user has two main roles which are not necessarily mutually exclusive. These are: \\
		\begin{enumerate}
		\item 		Advertiser: A user can place an ad (for free) on the homepage, which is then publicly visible.
		\item		Client: A client can look through the ads put up by advertisers and can contact the advertisers via the message functionality.
		\end{enumerate}
		
	\subsubsection{Premium User}
		By paying a fee for additional features, the premium user strongly increases the resources 
		of the programming team. The premium user gets notified immediately and can, in exchange for money, choose to have his ads placed on the home
		page of a-bec.
		
\subsection{Operating Environment}
	a-bec runs on any frequently used browser, i.e. Firefox, Safari, Chrome, Microsoft Edge, Internet Explorer.
	
\subsection{Design and Implementation Constraints}
	The template of a-bec was delivered using Java, MySQL, Hibernation and JavaScript. Changing this would 
	present unnecessary additional effort, which is why, in this way, we are constrained by these frameworks and languages.\\
	The software should be maintainable, so object-oriented design will be best-practice.
	Other than that, there seem to be no requirements.